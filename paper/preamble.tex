\documentclass[twoside,11pt]{article}

% Any additional packages needed should be included after obs_study_style.
% Note that obs_study_style.sty includes epsfig, amssymb, natbib and graphicx,
% and defines many common macros, such as 'proof' and 'example'.
%
% It also sets the bibliographystyle to plainnat; for more information on
% natbib citation styles, see the natbib documentation, a copy of which
\usepackage{obs_study_style}
\usepackage[utf8]{inputenc}
\usepackage{mathtools}
\usepackage{amsmath}
\usepackage{fancyvrb}
\usepackage[pdfstartview=FitH]{hyperref}
\usepackage{bookmark}

% Definitions of handy macros can go here
\providecommand{\tightlist}{%
  \setlength{\itemsep}{0pt}\setlength{\parskip}{0pt}
}
\newcommand{\dataset}{{\cal D}}
\newcommand{\fracpartial}[2]{\frac{\partial #1}{\partial  #2}}
\DefineVerbatimEnvironment{Sinput}{Verbatim}{fontshape=sl}
\DefineVerbatimEnvironment{Soutput}{Verbatim}{}
\DefineVerbatimEnvironment{Scode}{Verbatim}{fontshape=sl}
\newenvironment{Schunk}{}{}
\DefineVerbatimEnvironment{Code}{Verbatim}{}
\DefineVerbatimEnvironment{CodeInput}{Verbatim}{fontshape=sl}
\DefineVerbatimEnvironment{CodeOutput}{Verbatim}{}
\newenvironment{CodeChunk}{}{}
\setkeys{Gin}{width=0.8\textwidth}

% For papers submitted for review, just fill in author names
% For accepted papers, heading arguments are {volume}{year}{pages}{submitted}{published}{author-full-names}
\heading{}{}{}{}{}{David Benkeser and Nima S.~Hejazi}

% Short headings should be running head and authors last names

\ShortHeadings{\texttt{drtmle}: Doubly-robust Inference}{Benkeser and Hejazi}
\firstpageno{1}

\begin{document}

\title{Doubly-Robust Inference in R using \texttt{drtmle}}

\author{
  \name David Benkeser \email benkeser@emory.edu\\
  \addr Department of Biostatistics \& Bioinformatics\\
  Rollins School of Public Health\\
  Emory University\\
  1518 Clifton Road, N.E.,\\
  Atlanta, GA 30322
  \AND
  \name Nima S.~Hejazi \email nhejazi@hsph.harvard.edu\\
  \addr Department of Biostatistics\\
  T.H.~Chan School of Public Health\\
  Harvard University\\
  677 Huntington Avenue,\\
  Boston, MA 02115
}

\maketitle

\begin{abstract}
  Inverse probability of treatment weighted estimators and doubly-robust
  estimators (including augmented inverse probability of treatment weight and
  targeted minimum loss-based estimators) are widely used in causal inference to
  estimate and draw inference about the average effect of a treatment. As an
  intermediate step, these estimators require estimation of key nuisance
  parameters, which are often regression functions. Typically, regressions are
  estimated using maximum likelihood and parametric models. Confidence intervals
  and p-values may be computed based on standard asymptotic results, such as the
  central limit theorem, the delta method, and the nonparametric bootstrap.
  However, in high-dimensional settings maximum likelihood estimation often
  breaks down and standard procedures no longer yield correct inference.
  Instead, we may rely on adaptive estimators of nuisance parameters to
  construct flexible regression estimators. However, use of adaptive estimators
  poses a challenge for performing statistical inference about an estimated
  treatment effect. While doubly-robust estimators facilitate inference when
  *all* relevant regression functions are consistently estimated, the same
  cannot be said when at least one nuisance estimator is inconsistent.
  \texttt{drtmle} implements doubly-robust confidence intervals and hypothesis
  tests for targeted minimum loss-based estimates of the average treatment
  effect, in addition to several other recently-proposed estimators of the
  average treatment effect.
\end{abstract}

\begin{keywords}
  causal inference, machine learning, treatment effects, targeted minimum
  loss estimation
\end{keywords}
